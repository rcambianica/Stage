\documentclass{article}

\usepackage{hyperref}


\title{Protocollo FHIR (Fast Healthcare Interoperability Resources)}
\author{Riccardo Cambianica}

\begin{document}
    \maketitle
    \newpage
    \tableofcontents
    \newpage
    
    \section{Cos'è e a cosa serve}
    FHIR è uno standard di interoperabilità destinato a facilitare lo scambio di informazioni sanitarie tra operatori sanitari, pazienti, operatori sanitari, contribuenti, ricercatori e chiunque altro sia coinvolto nell'ecosistema sanitario.
    
    \section{Da cosa è composto}
         Si compone di 2 parti principali - un modello di contenuto in forma di "risorse", e una specifica per lo scambio di queste risorse in forma di interfacce RESTful in tempo reale, nonché messaggi e documenti. FHIR si basa a sua volta sul protocollo HL7 (Health Level 7). 
        Nello standard FHIR tutto è gestito come risorsa. Ad alto livello lo standard specifica cinque categorie principali di datatypes, da cui poi derivano le classi di dati specifiche, di seguito l'elenco.
        \begin{itemize}
            \item tipi astratti che fungono da base fondativa di i tipi, ad esempio le variabili standard di altri linguaggi di programmazione, come integer o string
            \item tipi semplici/primitivi, che nella pratica sono elementi singoli con un valore pr
        \end{itemize}
    
\end{document}